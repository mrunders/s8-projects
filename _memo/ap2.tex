\part{Apprentissage}
\pagebreak

\chapter{Approche par la logique}
Une approche simple concernant l'apprentissage de problèmes dont le domaine de sortie est boolean
serait de passer par la logique classique pour pouvoir simplifier la compréhension du problème.\\
\pagebreak
\section{Espace de Version}
Pour un problème suivant:

$\begin{tabular}{cccc|c}
\hline
Marque & couleur & nombre de places & transmition & accaptable? \\
\hline
Opel & gris & 5 & manuelle & oui\\
Ford & gris & 2 & manuelle & oui\\
Ford & rouge & 5 & automatique & oui\\
Renault & blue & 5 & automatique & non\\
Opel & rouge & 2 & manuelle & non\\
\hline
\end{tabular}$

\ \\
D'où il suffirait d'une fonction donnant dans le domaine Boolean, associer un algorithme de classification simple:

\begin{tikzpicture}
  \begin{axis} [
      xlabel     = X, % label x axis
      ylabel     = Y, % label y axis
      axis lines = left, %set the position of the axes
      clip       = false, 
      xmin       = 0,
      ymin       = 0,
    ]
    \addplot [color=black] coordinates { (0,0)(10,10) };
    \addplot [only marks, mark=*, color=blue] coordinates {(1,6)(6,8)(3,8) };
    \addplot [only marks, mark=*, color=red] coordinates {(9,1)(6,3) };
  \end{axis}
\end{tikzpicture}

Ayant comme points de couleurs $\crouge{Rouge}$ les points donnant la valeur de vérité False et les points de couleurs $\cblue{Blue}$ les point donnant la valeur de vérité True.\\
Mais ce ne serait pas donner un gros mode de résolution à un problème qui peut être simplifié?\\
\\
Pour les cas suivants:
\begin{description}
\item[-] Faciliter la compréhension du problème
\item[-] Comprendre pourquoi une décision donné pour une entrée
\end{description}
\pagebreak

\subsection{convergence des données}

\begin{tikzpicture}[->,>=stealth',shorten >=1pt,auto,node distance=2.8cm,
                    semithick]
  \tikzstyle{every state}=[fill=white,draw=none,text=black]

  \node[state]         (Z)                    {$\{\}$};
  \node[state]         (B) [below left  of=Z] {$B$};
  \node[state]         (A) [left  of=B]       {$A$};
  \node[state]         (C) [below right of=Z] {$C$};
  \node[state]         (D) [right of=C]       {$D$};
  \node[state]		   (AC) [below  of=A]     {$AC$};
  \node[state]		   (AB) [left of=AC]      {$AB$};
  \node[state]		   (AD) [right of=AC]     {$AD$};
  \node[state]         (BC) [right of=AD]     {$BC$};
  \node[state]		   (BD) [right of=BC]	  {$BD$};
  \node[state]		   (CD) [right of=BD] 	  {$CD$};
  \node[state]		   (ABD) [below of=AD]    {$ABD$};
  \node[state] 		   (ABC) [left of=ABD]    {$ABC$};
  \node[state]		   (ACD) [right of=ABD]   {$ACD$};
  \node[state]		   (BCD) [right of=ACD]   {$BCD$};
  \node[state]	       (ABCD) [below of=ABD]  {$ABCD$};
  

  \path (Z) edge              node {} (A)
            edge              node {} (B)
            edge			  node {} (C)
            edge  			  node {} (D)
        (A) edge			  node {} (AB)
        	edge			  node {} (AC)
        	edge			  node {} (AD)
        (B) edge			  node {} (AB)
        	edge			  node {} (BC)
        	edge 			  node {} (BD)
        (C) edge 			  node {} (AC)
        	edge			  node {} (BC)
        	edge 			  node {} (CD)
        (D) edge 			  node {} (AD)
        	edge 			  node {} (BD)
        	edge 			  node {} (CD)
        (AB) edge 			  node {} (ABC)
        	edge 			  node {} (ABD)
        (AC) edge			  node {} (ABC)
        	edge 			  node {} (ACD)
        (AD) edge 			  node {} (ABD)
        	edge			  node {} (ACD)
        (BC) edge			  node {} (ABC)
        	edge  			  node {} (BCD)
        (BD) edge			  node {} (ABD)
        	edge 			  node {} (BCD)
        (CD) edge 			  node {} (ACD)
        	edge			  node {} (BCD)
        (ABC) edge 			  node {} (ABCD)
        (ACD) edge 			  node {} (ABCD)
        (ABD) edge 			  node {} (ABCD)
        (BCD) edge 			  node {} (ABCD);
\end{tikzpicture}

\pagebreak