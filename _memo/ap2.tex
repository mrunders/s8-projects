\part{Apprentissage}
\pagebreak

\chapter{Approche d'apprentissage par la logique}
Une approche simple concernant l'apprentissage de problèmes dont le domaine de sortie est boolean
serait de passer par la logique classique pour pouvoir simplifier la compréhension du problème.\\
\pagebreak
\section{Espace de Version}
Pour un problème suivant:

$\begin{tabular}{cccc|c}
\hline
A & B & C & D & accaptable? \\
\hline
1 & 1 & 1 & 0 & oui\\
1 & 1 & 0 & 1 & oui\\
0 & 1 & 1 & 1 & non\\
\hline
\end{tabular}$

\ \\
D'où il suffirait d'une fonction donnant dans le domaine Boolean, associer un algorithme de classification simple:\\
\ \\
\vspace{1cm}
\scalebox{0.7}{
\begin{tikzpicture}
  \begin{axis} [
      xlabel     = X, % label x axis
      ylabel     = Y, % label y axis
      axis lines = left, %set the position of the axes
      clip       = false, 
      xmin       = 0,
      ymin       = 0,
    ]
    \addplot [color=black] coordinates { (0,0)(5,5) };
    \addplot [only marks, mark=*, color=blue] coordinates {(1,4)(3,3.5) };
    \addplot [only marks, mark=*, color=red] coordinates {(4,2) };
  \end{axis}
\end{tikzpicture}
}

\vspace{1cm}

Ayant comme points de couleurs $\crouge{Rouge}$ les points donnant la valeur de vérité False et les points de couleurs $\cblue{Blue}$ les point donnant la valeur de vérité True.\\
Mais ce ne serait pas donner un gros mode de résolution à un problème qui peut être simplifié?\\
\\
Pour les cas suivants:
\begin{description}
\item[-] Faciliter la compréhension du problème
\item[-] Comprendre pourquoi une décision donné pour une entrée
\end{description}
\pagebreak

\subsection{convergence des données}

Dans le tableau d'acceptation on peut transformé la règle 3 en son dual via la Lois de De Morgan:\\

\begin{multicols}{2}
[Et via un treillis de donnée pour chaque entré positif on peut compter le nombre d'occurrence de motif en faveur de l'acceptation de la ligne:]
\scalebox{0.3}{
\begin{tikzpicture}[->,>=stealth',shorten >=1pt,auto,node distance=2.8cm,
                    semithick]
  \tikzstyle{every state}=[fill=white,draw=none,text=black]

  \node[state]         (Z)                    {$\{\}$};
  \node[state]         (B) [below left  of=Z] {$B$};
  \node[state]         (A) [left  of=B]       {$A$};
  \node[state]         (C) [below right of=Z] {$C$};
  \node[state]         (D) [right of=C]       {$D$};
  \node[state]		   (AC) [below  of=A]     {$AC$};
  \node[state]		   (AB) [left of=AC]      {$AB$};
  \node[state]		   (AD) [right of=AC]     {$AD$};
  \node[state]         (BC) [right of=AD]     {$BC$};
  \node[state]		   (BD) [right of=BC]	  {$BD$};
  \node[state]		   (CD) [right of=BD] 	  {$CD$};
  \node[state]		   (ABD) [below of=AD]    {$ABD$};
  \node[state] 		   (ABC) [left of=ABD]    {$ABC$};
  \node[state]		   (ACD) [right of=ABD]   {$ACD$};
  \node[state]		   (BCD) [right of=ACD]   {$BCD$};
  \node[state]	       (ABCD) [below of=ABD]  {$ABCD$};
  

  \path (Z) edge              node {} (A)
            edge              node {} (B)
            edge			  node {} (C)
            edge  			  node {} (D)
        (A) edge			  node {} (AB)
        	edge			  node {} (AC)
        	edge			  node {} (AD)
        (B) edge			  node {} (AB)
        	edge			  node {} (BC)
        	edge 			  node {} (BD)
        (C) edge 			  node {} (AC)
        	edge			  node {} (BC)
        	edge 			  node {} (CD)
        (D) edge 			  node {} (AD)
        	edge 			  node {} (BD)
        	edge 			  node {} (CD)
        (AB) edge 			  node {} (ABC)
        	edge 			  node {} (ABD)
        (AC) edge			  node {} (ABC)
        	edge 			  node {} (ACD)
        (AD) edge 			  node {} (ABD)
        	edge			  node {} (ACD)
        (BC) edge			  node {} (ABC)
        	edge  			  node {} (BCD)
        (BD) edge			  node {} (ABD)
        	edge 			  node {} (BCD)
        (CD) edge 			  node {} (ACD)
        	edge			  node {} (BCD)
        (ABC) edge 			  node {} (ABCD)
        (ACD) edge 			  node {} (ABCD)
        (ABD) edge 			  node {} (ABCD)
        (BCD) edge 			  node {} (ABCD);
\end{tikzpicture}
}
$\begin{tabular}{cccc|c}
\hline
A & B & C & D & accaptable? \\
\hline
1 & 1 & 1 & 0 & oui\\
1 & 1 & 0 & 1 & oui\\
1 & 0 & 0 & 0 & oui\\
\hline
\end{tabular}$
\end{multicols}
\ \\

\begin{multicols}{2}
[]
\scalebox{0.4}{
\begin{tikzpicture}[->,>=stealth',shorten >=1pt,auto,node distance=2.8cm,
                    semithick]
  \tikzstyle{every state}=[fill=white,draw=none,text=black]

  \node[state]         (Z)                    {$\{\}$};
  \node[state]         (B) [below  of=Z]      {$B$};
  \node[state]         (A) [left  of=B]       {$A$};
  \node[state]         (C) [right of=B]       {$C$};
  \node[state]		   (AB) [below of=A]      {$AB$};
  \node[state]		   (AC) [right of=AB]     {$AC$};
  \node[state]         (BC) [right of=AC]     {$BC$};
  \node[state] 		   (ABC) [below of=AC]    {$ABC$};
  

  \path (Z) edge              node {} (A)
            edge              node {} (B)
            edge			  node {} (C)
        (A) edge			  node {} (AB)
        	edge			  node {} (AC)
        (B) edge			  node {} (AB)
        	edge			  node {} (BC)
        (C) edge 			  node {} (AC)
        	edge			  node {} (BC)
        (AB) edge 			  node {} (ABC)
        (AC) edge			  node {} (ABC)
        (BC) edge			  node {} (ABC);
\end{tikzpicture}
}
$\begin{tabular}{cccc|c}
\hline
A & B & C & D & accaptable? \\
\hline
\crouge{1} & \crouge{1} & \crouge{1} & \crouge{0} & \crouge{oui}\\
1 & 1 & 0 & 1 & oui\\
1 & 0 & 0 & 0 & oui\\
\hline
\end{tabular}$
\end{multicols}
\pagebreak

\begin{multicols}{2}
[]
\scalebox{0.4}{
\begin{tikzpicture}[->,>=stealth',shorten >=1pt,auto,node distance=2.8cm,
                    semithick]
  \tikzstyle{every state}=[fill=white,draw=none,text=black]

  \node[state]         (Z)                    {$\{\}$};
  \node[state]         (B) [below  of=Z]      {$B$};
  \node[state]         (A) [left  of=B]       {$A$};
  \node[state]		   (AB) [below of=A]      {$AB$};
  

  \path (Z) edge              node {} (A)
            edge              node {} (B)
        (A) edge			  node {} (AB)
        (B) edge			  node {} (AB);
\end{tikzpicture}
}
$\begin{tabular}{cccc|c}
\hline
A & B & C & D & accaptable? \\
\hline
1 & 1 & 1 & 0 & oui\\
\crouge{1} & \crouge{1} & \crouge{0} & \crouge{1} & \crouge{oui}\\
1 & 0 & 0 & 0 & oui\\
\hline
\end{tabular}$
\end{multicols}

\begin{multicols}{2}
[]
\scalebox{0.4}{
\begin{tikzpicture}[->,>=stealth',shorten >=1pt,auto,node distance=2.8cm,
                    semithick]
  \tikzstyle{every state}=[fill=white,draw=none,text=black]

  \node[state]         (Z)                    {$\{\}$};
  \node[state]         (A) [left  of=B]       {$A$};
  

  \path (Z) edge              node {} (A);
\end{tikzpicture}
}
$\begin{tabular}{cccc|c}
\hline
A & B & C & D & accaptable? \\
\hline
1 & 1 & 1 & 0 & oui\\
1 & 1 & 0 & 1 & oui\\
\crouge{1} & \crouge{0} & \crouge{0} & \crouge{0} & \crouge{oui}\\
\hline
\end{tabular}$
\end{multicols}

\ \\
Par itération et réduction du treillis on sait que $A$ et un attribut très discriminant, qui fait revenir le problème à seulement la valeur de $A$.\\

\pagebreak
\chapter{Apprentissage statistique}

Dans ce chapitre nous nous intéressons à des fonctions $h \in H$ à qui pour une liste $X$ à $d$ dimension de domaine réelle associe un label $y$ dans le domaine $[-1,+1]$. Un $x \in X$ peut être une couleur, un réelle, une chose négatif ou encore une mesure quelconque.
\pagebreak

\section{Classification binaire réalisable}

Chaque entrée $x \in X$ est tirée aléatoirement et indépendamment selon une distribution de probabilité $d$ qui est fixée mais inconnue de l'apprenant.
\\
Chaque sortie $y \in Y$ est calculé via la fonction cible $h* \in H$ qui est inconnue de l'apprenant.

\subsection{Erreur de généralisation et d'entrainement}

La performance d'une hypothèse $h \in H$ est calculé par le nombre d'erreurs que la fonction peut commettre en probabilité selon $d$:
\begin{description}
\item[] $l_d(h)$ = $P_{x~d}[h(x) \neq h*(x)]$
\end{description}

En pratique, l'apprenant n'a accès qu'a une petite partit nommé $S \in X$ (qui peut contenir des doublons) dont les éléments dont générés aléatoirement via $d$, Le risque empirique de $h$ par rapport à $S$ est donné par :
\begin{description}
\item[] $l_s(h)$ = $\frac{1}{|S|} |\{x \in S : h(x) \neq h*(x)\}|$
\end{description}
Le nombre d'erreur moyen que fait $h$ sur $S$\\

\subsection{Processus d'apprentissage}

Le processus d'apprentissage n'est pas si différent que dans la première partie du Memo:\\

Soit une distribution $d$, chaque requêtes vers $d$ va choisir un échantillons aléatoirement pour crée un ensemble $S$ qui va servir à faire apprendre $h$ lors de la phase d'apprentissage, tester lors de la phase de teste et retenir les erreur vies les fonction d'analyse.\\

\subsection{Incertitude de l'apprentissage}

Il existe deux mesures de l'incertitude en apprentissage statistique
\begin{description}
\item[] Paramètre de confiance qui donne la qualité de l'échantillonnage
\item[] Paramètre d'erreur qui donne un indice sur les bonnes prédictions futures
\end{description}

\subsection{Modèle PAC réalisable}
Une classe d(hypothèses $H$ est dite PCA (probability approximately correct) s'il existe une fonction $\{0,1\}^2 \rightarrow \{0,1,2.....\}$ telle que pour toute paire ($\phi$(confiance),$\psi$(erreur)) pour toute distribution $d$ sur $X$ et toute fonction cible $h* \in H$:
\begin{description}
\item[] Après avoir observé un échantillon $S$ de $X$ tiré aléatoirement selon $d$, et de taille au moins $m(\phi,\psi)$.
\item[] L'apprenant retourne une hypothèse $h \in H$, telle qu'avec une probabilité au moins $1 - \phi$, l'erreur de génération $l_d(h)$ est d'au plus $\psi$.
\end{description}

\section{Classes d'hypothèses finies}

Supposons $X$ = $[0,1]^d$

\begin{description}
\item[] Toutes fonction $h: [0,1]^d \rightarrow [0,1]$ est appelée fonction booléenne.
\item[] Une classe d'hypothèses booléennes est un sous ensemble $H$ de $[[0,1]^d \rightarrow [0,1]]$.
\end{description}

\subsection{Minimisation des erreurs empirique}
Le principe est de trouver dans $H$ l'hypothèse qui fait le moins d'erreurs sur l'échantillon $S$:
\formula{$h_S \in argmin L_S(h), h \in H$}


\subsection{Théorème de PAC des classes finies}
Toutes classe d'hypothèse $H$ finie est PAC-apprenable avec une complexité d'échantillonnage
\formula{$m(\phi,\psi) \leq \frac{ln(|H|/\phi)}{\psi}$}

\pagebreak
\section{Classification binaire agnostique}
\subsection{Régression agnostique}

\pagebreak
\chapter{Apprentissage Online}

L'apprentissage online est un jeu à somme nulle répétitif à deux joueur (théorème minmax ou équilibre de Nash), les joueurs sont l'environnement et le joueur.\\
L'apprenant reçoit une observation de l'environnement et donne une prédiction, et l'environnement va donner la vérification sur la prédiction.\\

\pagebreak
\section{Analyse convexe}
\subsection{Combinaison convexe}

Une forme convexe est une forme pour qui n'importe quel droite dont les 2 points font partie de la forme est dans la forme.\\
%% fill convex true

\subsection{Convex Hull}

%% fill l'enveloppe de la forme

\subsection{Convex Set}

%% fill

%% si espace d'app est convex, alors pb is polynomial. sinon np dur

\subsection{Theoreme de la séparation des hyperplan}

%% fill

\subsection{Gradient}

%% fill

\subsection{Fonction de Perte}

%% fill

\section{Apprentissage par régression}

\section{Apprentissage par classification}

\pagebreak

