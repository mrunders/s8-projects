\part{Problème de satisfaction de contraintes CSP}
\pagebreak

\chapter{Introduction et modèles}
\pagebreak

Une Solution d'un problème CSP est une assignation d'une valeur à chaque variables de P tel que toutes les contraintes de P soit satisfaite.\\

\section{exemple simple}

Trouver une assignation Minimal et Maximal pour chaque variables de P dans le problème suivant:\\
\begin{description}
\item[vars(P)] = $w,x,y,z$
\item[Dom(P)]\  \begin{description}
\item[Dom(w,x)] = $\{1,2,3\}$
\item[Dom(y,z)] = $range(4)$
\end{description}
\item[Contraintes]\  \begin{description}
\item[$~$] $w = x$
\item[$~$] $x \leq y + 1$
\item[$~$] $y > z$
\item[$~$] $(x,z) \in \{(1,2),(2,1),(2,4),(3,3)\}$
\end{description}
\end{description}
\ \\
Une solution serait:
\begin{description}
\item[Minimal] $w=x=1, y=3, z=2$
\item[Maximal] $w=x=z=3, y=4$
\end{description}

\pagebreak
\subsection{Graphe de contraintes}

Chaque variable est un nœud et chaque contraintes est représenté par une arête entre les variables concerné.\\
\begin{center}
\begin{tikzpicture}[-,>=stealth',shorten >=1pt,auto,node distance=2.8cm,
                    semithick]
  \tikzstyle{every state}=[fill=white,draw=none,text=black]

  \node[state]         (W)                    {$W$};
  \node[state]         (X) [left of=W]        {$X$};
  \node[state]         (Y) [below right of=X] {$Y$};
  \node[state]		   (Z) [below left of=X]  {$Z$};
  

  \path (X) edge              node {} (W)
            edge              node {} (Y)
            edge			  node {} (Z)
        (Z) edge			  node {} (Y);
\end{tikzpicture}
\end{center}
\subsection{Graphe de compatibilité}

Représenter toutes variables avec des ensemble contenant toutes les valeur de leurs domaine puis relier chaque éléments de l'ensemble avec un autre tel que les contraintes ne soit pas violet:\\
\begin{center}
\includegraphics[scale=0.6]{img/aic_csp_1.jpg} 
\end{center}

\chapter{Filtrage}
\pagebreak
\section{Filtrage du domaine via les contraintes}

Un problème peut être définit comme une intersection de sous problème à qui un sous problème est représenté par un filtre du domaine des variable et une contrainte qui va réduire le domaine des variable à un étant de consistance peu importe les valeurs des variables.\\

\begin{description}
\item[Arc Consistency (AC)] Tous les valeurs inconsistant sont identifié et retiré
\item[Bound Consistency (BC)] Les valeurs inconsistant sont les bornes des domaine et ils sont identifié et retiré
\end{description}

\subsection{Exemple}

\formula{Contrainte $w + 3 = z$}
Avec
\formula{$dom(w) = \{1,3,4,5\}$}
\formula{$dom(z) = \{4,5,8\}$}

Avec un filtre AC:\\
\formula{$dom(w) = \{1,5\}$ et $dom(z) = \{4,8\}$}

Avec un filtre BC:\\
\formula{$dom(w) = \{1,3,4,5\}$ et $dom(z) = \{4,5,8\}$}

\pagebreak
\section{Notion de Support}
Soit $c_{xyz}$ une contrainte tel que $c_{xyz}$ égal:
\formula{$dom(x,y) = \{a,b\} et dom(z) = \{b,c\}$}

On n'a $T = rel(c_{xyz})$ et V = $dom(x)$ x $dom(y)$ x $dom(z)$:\\

\begin{multicols}{2}
[$\cvert{(z,b)}$ a un support mais $\crouge{(z,c)}$ n'en n'a pas]

\begin{tabular}{|ccc|}
\hline $ $ & T & $ $\\
\hline
a & a & a \\
$\cvert{a}$ & $\cvert{b}$ & $\cvert{b}$\\
$\crouge{a}$ & $\crouge{c}$ & $\crouge{c}$\\
 b & a & a \\ b & b & b \\
c & a & a \\
$\crouge{c}$ & $\crouge{c}$ & $\crouge{c}$\\
\hline
\end{tabular}

\begin{tabular}{|ccc|}
\hline $ $ & V & $ $\\
\hline
a&a&b\\
$\crouge{a}$&$\crouge{a}$&$\crouge{c}$\\
$\cvert{a}$&$\cvert{b}$&$\cvert{b}$\\
$\crouge{a}$&$\crouge{b}$&$\crouge{c}$\\
b&a&b\\
$\crouge{b}$&$\crouge{a}$&$\crouge{c}$\\
b&b&b\\
$\crouge{b}$&$\crouge{b}$&$\crouge{c}$\\
\hline
\end{tabular}
\end{multicols}
\pagebreak
\section{AC filtre AllDifferent}

\begin{multicols}{2}
[Soit un sudoku block avec la contrainte $allDifferent(w,x,y,z)$:]
\begin{tabular}{|c|c|c|}
\hline $\cvert{3}$ & w & $\cvert{6}$\\
\hline x & $\cvert{8}$ & y\\
\hline $\cvert{1}$ & $\cvert{4}$ & z\\
\hline
\end{tabular}
\begin{description}
\item[w] = \{2,5\}
\item[x] = \{5,7,9\}
\item[y] = \{7\}
\item[z] = \{2,5\}
\end{description}
\end{multicols}

\begin{multicols}{2}
[Dans ce cas, y contient un singleton, donc il sera trivialement résolut en affectant y = 7 et en suppriment tout les occurrences de 7 dans les autres variables:]
\begin{tabular}{|c|c|c|}
\hline 3 & w & 6\\
\hline x & 8 & $\cvert{7}$\\
\hline 1 & 4 & z\\
\hline
\end{tabular}
\begin{description}
\item[w] = \{2,5\}
\item[x] = \{5,$\crouge{7}$,9\}
\item[z] = \{2,5\}
\end{description}
\end{multicols}

\begin{multicols}{2}
[nous pouvons construire le Hall sets suivant: $(w,z) vs (2,5)$ et donc supprimer toutes les occurrences de 2 et 5 dans les autres variables:]
\begin{tabular}{|c|c|c|}
\hline 3 & w & 6\\
\hline $\cvert{9}$ & 8 & 7\\
\hline 1 & 4 & z\\
\hline
\end{tabular}
\begin{description}
\item[w] = \{2,5\}
\item[x] = \{$\crouge{5}$,9\}
\item[z] = \{2,5\}
\end{description}
\end{multicols}

\begin{multicols}{2}
[Ce qui nous donne 2 solutions, (w=2,z=5) ou (w=5,z=2)]
\begin{tabular}{|c|c|c|}
\hline 3 & w & 6\\
\hline 9 & 8 & 7\\
\hline 1 & 4 & z\\
\hline
\end{tabular}
\begin{description}
\item[w] = \{2,5\}
\item[z] = \{2,5\}
\end{description}
\end{multicols}
\pagebreak

\begin{multicols}{2}
[Soit un autre exemple plus complexe avec une contrainte $allDifferent(u,v,w,x,y,z)$ et un domaine $\{1,2,5,6,7,9\}$:]
\begin{description}
\item[u] = \{1,2\}
\item[v] = \{2,9\}
\item[w] = \{1,5,6\}
\item[x] = \{2,6\}
\item[y] = \{6,7,9\}
\item[z] = \{1,9\}
\end{description}
\end{multicols}

\begin{multicols}{2}
[On peut crée une Hall sets de taille 3 $(u,v,z)=(1,2,9)$ et éliminer toutes les occurrences de 1,2,9 dans les autres variables:]
\begin{description}
\item[u] = \{1,2\}
\item[v] = \{2,9\}
\item[w] = \{5,6\}
\item[x] = \{6\}
\item[y] = \{6,7\}
\item[z] = \{1,9\}
\end{description}
\end{multicols}

\begin{multicols}{2}
[On peut affecter x à 6 et supprimer toutes les occurrences de 6 dans les autres variables:]
\begin{description}
\item[u] = \{1,2\}
\item[v] = \{2,9\}
\item[w] = 5
\item[x] = 6
\item[y] = 7
\item[z] = \{1,9\}
\end{description}
\end{multicols}

\begin{multicols}{2}
[On obtient donc la tableau suivant qui pour n'importe quelle valeurs de u,v,w,x,y,z donne:]
\begin{tabular}{cccccc}
u&v&w&x&y&z\\ \hline 1&2&5&6&7&1\\ 2&9&$ $&$ $&$ $&9\\
\end{tabular}
\end{multicols}
\pagebreak
\section{AC filtre Cardinalité}

Une contrainte de cardinalité note $cardinality(X,Y,L,U)$ qui a $X$ est l'ensemble des variables, $Y$ le domaine des variables de $X$, $(L_i,U_i)$ le range d'occurrences de la variable $Y_i$ dans tout le modèle.\\
Si $(L_i=1,U_i=3)$ alors la variable $Y_i$ ne pourra être présent de 1 à 3 fois maximum, si $(L_i=2,U_i=2)$ alors la variable $Y_i$ devra être présent 2 fois.\\

Prenons:
\begin{description}
\item[X] les agents \{Peter,Paul,Mary,John,Bob,Mike,Julia\}
\item[Y] les activités \{Morning (M),Day (D),Night (N),Backup (B),Off (O)\}
\item[L] \{1,1,1,0,0\}
\item[U] \{2,2,1,2,2\}
\end{description}

Nous voudrons réduire ce genre de tableau suivant:\\
\begin{tabular}{c|ccccccc}
$ $&Monday&Tuesday&Wendsday&Thursday&Friday&Saturday&Sunday\\ \hline
Peter&D&N&N&N&O&O&O\\
Paul&O&O&D&D&M&M&B\\
Mary&M&M&D&D&O&O&N\\
\end{tabular}

\begin{center} \scalebox{0.8}{
\begin{tikzpicture}[-,>=stealth',shorten >=1pt,auto,node distance=2cm,
                    semithick]
  \tikzstyle{every state}=[fill=white,draw=none,text=black]

  \node[state]         (PETER)                  {$Peter$};
  \node[state]         (PAUL) [right of=PETER]  {$Paul$};
  \node[state]         (MARY) [right of=PAUL]   {$Mary$};
  \node[state]         (JOHN) [right of=MARY]   {$John$};
  \node[state]         (BOB)  [right of=JOHN]   {$Bob$};
  \node[state]         (MIKE) [right of=BOB]    {$Mike$};
  \node[state]         (JULIA)[right of=MIKE]   {$Julia$};
  \node[state]         (NONE)[below right of=PETER] {$ $};
  \node[state]         (M)[below right of=NONE]{$M(1,2)$};
  \node[state]         (D)[right of=M]          {$D(1,2)$};
  \node[state]         (N)[right of=D]          {$N(1,1)$};
  \node[state]         (B)[right of=N]          {$B(0,2)$};
  \node[state]         (O)[right of=B]          {$O(0,2)$};

  \path (M) edge node {} (PETER)
            edge node {} (PAUL)
            edge node {} (MARY)
            edge node {} (JOHN)
            edge node {} (BOB)
        (D) edge node {} (PETER)
            edge node {} (PAUL)
            edge node {} (JOHN)
            edge node {} (BOB)
            edge node {} (MIKE)
        (N) edge node {} (BOB)
            edge node {} (MIKE)
            edge node {} (JULIA)
        (B) edge node {} (MIKE)
        (O) edge node {} (MIKE)
            edge node {} (JULIA);
\end{tikzpicture}}
\end{center}

On peut assigner $Bob$ à la tache $N$ et assigner à $Julia$ la tache $O$, le reste sera des combinaison possible:
\begin{center} \scalebox{0.8}{
\begin{tikzpicture}[-,>=stealth',shorten >=1pt,auto,node distance=2cm,
                    semithick]
  \tikzstyle{every state}=[fill=white,draw=none,text=black]

  \node[state]         (PETER)                  {$Peter$};
  \node[state]         (PAUL) [right of=PETER]  {$Paul$};
  \node[state]         (MARY) [right of=PAUL]   {$Mary$};
  \node[state]         (JOHN) [right of=MARY]   {$John$};
  \node[state]         (BOB)  [right of=JOHN]   {$Bob$};
  \node[state]         (MIKE) [right of=BOB]    {$Mike$};
  \node[state]         (JULIA)[right of=MIKE]   {$Julia$};
  \node[state]         (NONE)[below right of=PETER] {$ $};
  \node[state]         (M)[below right of=NONE]{$M(1,2)$};
  \node[state]         (D)[right of=M]          {$D(1,2)$};
  \node[state]         (N)[right of=D]          {$N(1,1)$};
  \node[state]         (B)[right of=N]          {$B(0,2)$};
  \node[state]         (O)[right of=B]          {$O(0,2)$};

  \path (M) edge node {} (PETER)
            edge node {} (PAUL)
            edge node {} (MARY)
            edge node {} (JOHN)
        (D) edge node {} (PETER)
            edge node {} (PAUL)
            edge node {} (JOHN)
        (N) edge node {} (BOB)
        (B) edge node {} (MIKE)
        (O) edge node {} (MIKE)
            edge node {} (JULIA);
\end{tikzpicture}}
\end{center}

\section{AC filtre Sum à une borne}

Soit un filtre $sum$ avec les variables $(x,y,z,a,b,c)$ dans le domaine $range(1,5)$ appliqué sur l'équation:
\formula{$2x+3y+2z+a+4b+2c \geq 50$}
\begin{center}
\begin{tabular}{cccccc}
x&y&z&a&b&c\\ \hline
1&1&1&1&1&1\\ 2&2&2&2&2&2\\ 3&3&3&3&3&3\\ 4&4&4&4&4&4\\
\end{tabular}
\end{center}
Si le somme des maximums de chaque variable ne respecte pas la contrainte, alors il n'existe aucun assignement des variables tel que la contrainte serait satisfaite:
\formula{$2*4+3*4+2*4+4+4*4+2*4 = 8+12+8+4+16+8 = 56 \geq 50$}
Pour chaque variables, enlever son impacte dans le résultat précédemment obtenue et refaire l'opération de dépilé avec les différents $x$ tant que un $x_i$ ne satisfait pas la contrainte:\\
\ \\
Pour x:
\begin{description}
\item[] $2*4+3*4+2*4+4+4*4+2*4 = 8+12+8+4+16+8 = 56 \geq 50$
\item[] $ 56 - 2*4 + 2*1 = 50 \geq 50$, donc $x=1$ fonctionne
\item[] (pareil pour z,c)
\end{description}
\ \\
Pour y:
\begin{description}
\item[] $ 56 - 3*4 + 3*1 = 44 \geq 50$, donc $y=1$  ne fonctionne pas
\item[] $ 56 - 3*4 + 3*2 = 50 \geq 50$, donc $y=2$  fonctionne
\end{description}
\pagebreak
Pour b:
\begin{description}
\item[] $ 56 - 4*4 + 4*1 = 40 \geq 50$, donc $b=1$  ne fonctionne pas
\item[] $ 56 - 4*4 + 4*2 = 44 \geq 50$, donc $b=2$  ne fonctionne pas
\item[] $ 56 - 4*4 + 4*3 = 50 \geq 50$, donc $b=3$  fonctionne
\end{description}
\ \\
On obtient:\\
\begin{center}
\begin{tabular}{cccccc}
x&y&z&a&b&c\\ \hline
1&$ $&1&1&$ $&1\\ 2&2&2&2&$ $&2\\ 3&3&3&3&3&3\\ 4&4&4&4&4&4\\
\end{tabular}
\end{center}

\pagebreak
\section{AC filtre avec Sum à deux bornes}

\pagebreak