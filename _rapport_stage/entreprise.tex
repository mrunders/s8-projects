
\chapter{Contexte du Stage}
Dans ce chapitre, nous allons introduire l'entreprise d'accueil, le cadre de travail, une petite description des outils utilisés (seulement les outils qui ne méritent pas une trop grande intention), une description plus approfondit de certains outils dans le chapitre suivant, de la méthodologie de travail utilisé puis le sujet de la mission.
\pagebreak

\section{Présentation de l'entreprise}
Crée en 2005 par deux frères, l'entreprise $e$ maintenant basé à $q$, Étant une société de service, proposant à ses collaborateurs une solution humaine de personnes maîtrisant leurs sujets le tout pour donner une équipe opérationnelle pour toutes réalisations de projets.\linebreak
Les domaines de compétences vont du classique Java, passant par des standards comme php, docker vers l'incontournable Python. Aujourd'hui nous comptons environ une quarantaine de prestataires de talent couvrant les grands standards du monde numérique dans le quel nous baignons.\linebreak

\section{Présentation de la mission}
La demande de l'entreprise est de pouvoir avoir un service pouvant traiter les demandes des clients dans les heures où le standard d'écoute n'est présent, avant ce projet, toutes les demandes passèrent dans la catégorie des astreintes. Notre intelligence artificielle doit simuler une conversation normal tout en dirigent les questions posées par rapport à la problématique du client puis de décider si oui ou non la résultante de l'appelle sera une astreinte ou non.
\pagebreak

\section{Cadre de travail}
Le cadre de travail ce situe au Centre de service, dans un open-space assez spacieux, accompagné d'une petite cuisine, et de quelque salles de repos.\\
Les interactions se faisaient soit par Chat écrit, par mail ou via les réunions presque hebdomadaires qui constituait une partie de la méthodologie agile auquel j'étais impliqué dans les projets.
Bien entendu, à ne pas oublier le Git de l'entreprise pour toutes les solutions de travail collaboratif.

\section{Outils de travail}
Je vais énumérer les outils utilisées qui ont soit servi à l'analyse des data, la construction des jeux de données, les technologies vus qui n'ont pas étais implémenté et les algorithmes qui ont été testé.\\
\linebreak
\begin{description}
\item[DataIku]: Un logiciel qui permet le traitement de données de masse via de nombreux algorithmes présent par défaut allant de la simple transformation de dates (transformation, extraction, opération, ...) en traitement du langage naturel (lower, normalise, tokenise, ...), quelques algorithmes provenant du module python \textit{scikit-learn} comme la régression logistique, foret d'isolation ou un réseau neuronal permettant d'avoir une pré-visualisation de grands algorithmes, ceci a donné lieu à une simplification du choix des algorithmes pour les prédictions.
\item[DeepSpeech]: une solution de transformation de la voix en texte proposé par la fondation Mozilla utilisant la librairie python \textit{Tensorflow}.
\item[Spacy]: Une librairie python traitent des motifs du langage naturel, l'extraction de sous chaîne utilise la notion de \textit{POST-tagging} et ainsi découper le texte en une classe d'éléments tels que son genre, son lien avec le reste de la phrase, ...
\item[Apriori]: un algorithme de fouille de données qui se base sur la notion d'\textit{itemset}, cet algorithme a été abandonné même si dans un premier temps il donnait de meilleurs résultats que les réseaux bayésiens naïf lors de la partie ontologie.
\end{description}

\pagebreak
