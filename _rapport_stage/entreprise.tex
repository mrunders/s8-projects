
\chapter{Contexte du Stage}
Dans ce chapitre, nous allons introduire l'entreprise d'accueil, le cadre de travail, une petite description des outils utilisé (seulement les outils qui ne méritent pas une trop grande intention), une descriptions plus approfondit ce certains outils dans le chapitre suivant, de la méthodologie de travail utilisé puis le sujet de la mission.
\pagebreak

\section{Présentation de l'entreprise}
Crée en 2005 par deux frères, l'entreprise $\_$ maintenant basé à $\_$, Étant une société de service, proposant à ses collaborateurs une solution humaine de personnes maîtrisant leurs sujets le tout pour donner une équipe opérationnel pour toutes réalisation de projets.\linebreak
Les domaines de compétences vont du classique Java, passant par des standards comme php, docker vers l'incontournable Python. Aujourd'hui nous comptons environs une quarantaine de prestataire de talent couvrant les grand standards du monde numérique dans le quel nous baignons.\linebreak

\section{Présentation de la mission}
La demande du client est de pouvoir avoir un service pouvant traiter les demandes des clients dans les heurs où le standard d'écoute n'est présent, avant ce projet, toutes les demandes passèrent dans la catégorie des astreintes. Notre Intelligence artificiel doit simuler une conversation normal tout en dirigent les questions posé par rapport à la problématique du client puis de décider si oui ou non la résultante de l'appelle sera une astreinte ou non.
\pagebreak

\section{Cadre de travail}
Le cadre de travail ce situer au Centre de service, dans un open-space assez spacieux, accompagné d'une petit cuisine, et de quelques sales de repos.\\
Les interactions se faisais soit par Chat écrit, par mail ou via les réunions presque hebdomadaire qui constituai une partie de la méthodologie agile au quel j'étais impliqué dans les projets.
Bien entendu, à ne pas oublier le Git de l'entreprise pour toutes les solutions de travail collaboratif.

\section{Outils de travail}
Je vais énumérer les outils utilisé qui ont soit servit à l'analyse des data, la construction des jeux de données, les technologies vu qui n'ont pas étais implémenté et les algorithmes qui ont était testé.\\
\linebreak
\begin{description}
\item[DataIku]: Un logiciel qui permet le traitement de données de masse via de nombreux algorithmes présent par défaut allant de la simple transformation de dates (transformation, extraction, opération, ...) en traitement du langage naturel (lower, normalise, tokenise, ...), quelques algorithmes provenant du module python \textit{scikit-learn} comme les régressions logistique, foret d'isolation ou un réseau neuronal permettant d'avoir une prévisualisation des grands algorithmes dans une seul fenêtre, ceci a donné lieu à une simplification du choix des algorithmes pour les prédictions.
\item[DeepSpeech]: une solution de transformation de la voix en texte proposé par la fondation Mozilla utilisant la librairie python \textit{Tensorflow}.
\item[Spacy]: Une librairie python traitent des motifs du langage naturel, l'extraction de sous chaîne utilise la notion de \textit{POST-tagging} et ainsi découper le texte en une class d'éléments tel que son genre, son lien avec le reste de la phrase, ...
\item[Apriori]: un algorithme de fouille de donné qui se base sur la notion d'\textit{itemset}, cette algorithme a était abandonné même si dans un premier temps donnait de meilleurs résultats que le réseaux bayésiens naïf lors de la partie ontologie.
\end{description}

\pagebreak