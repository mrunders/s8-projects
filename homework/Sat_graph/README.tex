	\documentclass[french, 12pt]{report}
\usepackage[latin1, utf8]{inputenc}
\usepackage{color}
\usepackage{graphicx}
\usepackage{listings}
\usepackage{amssymb}
\usepackage{amsmath}
\usepackage{pstricks}
\usepackage{enumitem}
\usepackage{multicol}
\usepackage{verbatim}
\usepackage{listings}
\usepackage{tikz}
\usetikzlibrary{arrows,automata}
\usetikzlibrary{shapes,snakes}
\usepackage{pgfplots}
\usepackage{pgfplotstable}
\usepackage{xifthen}
\usepackage{fancyhdr}
\usepackage{caption}
 
%%\pgfplotsset{compat=1.16}
\setlist[description]{leftmargin=\parindent,labelindent=\parindent}

\definecolor{gray}{rgb}{0.4,0.4,0.4}
\definecolor{darkblue}{rgb}{0.0,0.0,0.6}
\definecolor{cyan}{rgb}{0.0,0.6,0.6}
\definecolor{darkgreen}{RGB}{0,150,0}
 	
\newcommand{\cblue}[1]{ \textcolor{blue}{#1}}
\newcommand{\corange}[1]{ \textcolor{orange}{#1}}
\newcommand{\cviolet}[1]{ \textcolor{violet}{#1}}
\newcommand{\crouge}[1]{ \textcolor{red}{#1}}
\newcommand{\cvert}[1]{ \textcolor{darkgreen}{#1}}
\newcommand{\cgris}[1]{ \textcolor{gray}{#1}}

%% ------------------------- Header Footer
\pagestyle{fancy}
\fancyhf{}
\fancyhead[LE,RO]{Master 2 IA Memo}
\fancyhead[RE,LO]{ }
\fancyfoot[CE,CO]{\leftmark}
\fancyfoot[LE,RO]{\thepage}
% -------------------------- END Header Footer

%% ------------------------- Formular
\newcommand{\cformular}[2]{
\begin{center} 
\begin{description} 
\item[#1] #2 
\end{description} 
\end{center}
}
%% ------------------------- END Formular

%% ------------------------- RO Model
\newcommand{\rovarinout}[5]{
\begin{description}
\item[La variable entrante sera] #1
\item[La variable sortante sera] #2 car:
\end{description}
\begin{multicols}{3}
#3, #4, #5
\end{multicols}
}

\newcommand{\romodel}[8]{
\begin{multicols}{2}
[Voici le nouveau modèle:]
\begin{description}
\item[Déterminer] #1
\item[#2] #3 % ((2) maximisant | minimisant )
\item[Variables hors base] #4
\item[Variables de Base] #5
\item[Solution admissible] #6 et Z = #7
\end{description}
#8 % contraintes
\end{multicols}
}
%% ------------------------- END RO Model

%% ------------------------- XML

\lstset{
  basicstyle=\ttfamily,
  columns=fullflexible,
  showstringspaces=false,
  commentstyle=\color{gray}\upshape
}

\lstdefinelanguage{XML}
{
  morestring=[b]",
  morestring=[s]{>}{<},
  morecomment=[s]{<?}{?>},
  stringstyle=\color{black},
  identifierstyle=\color{darkblue},
  keywordstyle=\color{cyan},
  morekeywords={xmlns,version,type}% list your attributes here
}
%% ------------------------ END XML

%% ------------------------ Almost all
\newcommand{\almost}{\mid\kern-0.40em{\backsim}\ }
%% ------------------------ END Almost all

%% ------------------------ Inverse DL lite
\newcommand{\inverse}{\urcorner\ }
%% ------------------------ End Inverse DL lite

%% ------------------------ Python code for Machine leaning
\definecolor{codegreen}{rgb}{0,0.6,0}
\definecolor{codegray}{rgb}{0.5,0.5,0.5}
\definecolor{codepurple}{rgb}{0.58,0,0.82}
\definecolor{backcolor}{rgb}{0.97,0.97,0.95}

\lstdefinestyle{mlpythoncode}{
    backgroundcolor=\color{backcolor},   
    commentstyle=\color{codepurple},
    keywordstyle=\color{codegreen},
    numberstyle=\tiny\color{codegray},
    stringstyle=\color{magenta},
    basicstyle=\footnotesize,
    breakatwhitespace=false,         
    breaklines=true,                 
    captionpos=b,                    
    keepspaces=true,                 
    numbers=left,                    
    numbersep=5pt,                  
    showspaces=false,                
    showstringspaces=false,
    showtabs=false,                  
    tabsize=2,   
    emph={[2]sklearn, model_selection, train_test_split,KFold, linear_model, LinearRegression, LogisticRegression,
    DecisionTreeRegressor, tree, neighbors, KNeighborsClassifier, svm, SVC, metrics, confusion_matrix,precision_recall_fscore_support,
    LeaveOneOut},
	emphstyle=[2]\color{blue}
}

\lstdefinestyle{myjson}{
    string=[s]{"}{"},
    stringstyle=\color{blue},
    comment=[l]{:},
    commentstyle=\color{black},
}

\newcommand{\sepline}{\textcolor{gray}{\noindent\rule{14cm}{0.1pt}}}
\newcommand{\paramtype}[1]{\textcolor{gray}{\textsf{\textit{#1}}}}

\newcommand{\funcdoc}[4]{
	\ \\
	\textit{\textsf{\cblue{#1}}}
    \ifthenelse{\isempty{#2}}%
    {}%
	{    \ \\\sepline\ \\
	\textbf{Variables}
	{#2}}
    \ifthenelse{\isempty{#3}}%
    {}%
	{    \ \\\sepline\ \\
	\textbf{Contraintes}
	{#3}}
    \ifthenelse{\isempty{#4}}%
    {}%
	{    \ \\\sepline\ \\
	\textbf{}
	{#4}}
}

%% ------------------------ END Python code

%% ------------------------ FORMULA
\newcommand{\formula}[1]{
\begin{center}
{#1}
\end{center}
}
%% ------------------------ END FORMULA


%% ------------------------ FORME SHAPE
\newcommand{\cshape}[2]{
\begin{center}
\scalebox{#1}{#2}
\end{center}
}
%% ----------------------- END FORME SHAPE

\title{Coloration de graph avec SAT}
\author{LAURENT Thomas}
\date{Master 2 informatique 2019}

\begin{document}
\maketitle
\pagebreak

\section{Ce que contient l'archive}

\begin{description}
\item[\_\_main\_\_.py] Le programme qui va parser les graphs en format DIMACS.
\item[model\_printer.py] Va correctement formater et afficher le modèle obtenue via le solveur SAT.
\item[tmp.json] Un fichier json contenant un graph (notamment celui de l'Australie).
\item[sat\_graph.sh] Pour automatiser le déroulement des algorithmes.
\item[README.pdf] Coucou c'est moi.
\end{description}

\section{Format du graph dans le fichier json}

\lstset{style=myjson}
\begin{lstlisting}
{
    "variables" : ["wa","nt","sa","gld","nsw","vc","tas"],
    "constraints": [["wa","nt"],["wa","sa"],...,["nsw","vc"]]
}
\end{lstlisting}

\begin{description}
\item[variables] étant une liste de tout les noms des points présent dans le graph.
\item[constraints] la représentation de toutes les arêtes du graph.
\end{description}

\pagebreak
\section{Utilisation du script}
Voici un exemple:
\formula{$./sat\_graph\ \ MiniSat\_v1.14\_linux\ \ tmp.json\ \ 3$}

\begin{description}
\item[MiniSat\_v1.14\_linux] Est le solveur utilisé en question MiniSat.
\item[tmp.json] Un fichier au format json valide et étant similaire à l'exemple du dessus.
\item[3] étant le nombre de couleurs que vous voudrez donner à l'algo. (il n'y a pas vérification si négatif si non numérique)
\end{description}

\ \\
Le script, sat et python vont générer deux fichiers lors de leurs exécution:

\begin{description}
\item[sat.in] Qui sera le fichier formaté en DIMACS qui est créer par python et lu par le solveur. (le fichier contient 2 lignes de commentaires indiquant le noms des variables et leurs ids, ceci pour aider le $model\_printer$).
\item[sat.out] Qui sera le modèle de base qui est écrit par le solveur et lu par le $model\_printer$ (je ne garantit pas que tout les solveur pourront être lu par le $model\_printer$).
\end{description}

\pagebreak

\section{Description du Modèle}

L'explication du choix de la modélisation se fera via l'exemple ci dessous pour le point nommé $B$ et pour un nombre de couleurs maximum de $3$:

\begin{tikzpicture}[-,>=stealth',shorten >=1pt,auto,node distance=2.8cm,semithick]
  \tikzstyle{every state}=[fill=white,draw=none,text=black]

  \node[state] 		   (A)                    {$A$};
  \node[state]         (B) [above right of=A] {$\cblue{B}$};
  \node[state]         (C) [below right of=B] {$C$};

  \path (B) edge              node {} (A)
            edge              node {} (C);
\end{tikzpicture}

\funcdoc{Soit $B$ le point concerné}{
\begin{description}
\item[] $A_1,A_2,A_3,B_1,B_2,B_3,C_1,C_2,C_3$, les 9 variables boolean des points $A,B,C$ tel que pour chaque couleurs lui associe un indice.
\item[] Si $B_3 \models \top$ alors $B$ prendra la couleur d'indice $3$.
\end{description}
}{
\begin{description}
\item[$B$] doit être composé d'au moins une couleur: $(B_1 \vee B_2 \vee B_3)$
\item[$B_i$] ne doit avoir d'une sauf affectation à $\top$ : $(\neg B_1 \vee \neg B_2) \wedge (\neg B_2 \vee \neg B_3) \wedge (\neg B_1 \vee \neg B_3)$
\item[le point $A$ doit avoir une couleur différente de celle de $B$] $(\neg A_1 \vee \neg B_1) \wedge (\neg A_2 \vee \neg B_2) \wedge (\neg A_3 \vee \neg B_3)$
\item[le point $C$ doit avoir une couleur différente de celle de $B$] $(\neg C_1 \vee \neg B_1) \wedge (\neg C_2 \vee \neg B_2) \wedge (\neg C_3 \vee \neg B_3)$
\end{description}
}{$ $}

\pagebreak

\section{Exemples}
Soit les deux exemples suivant sur le graph suivant:

\begin{tikzpicture}[-,>=stealth',shorten >=1pt,auto,node distance=2.8cm,semithick]
  \tikzstyle{every state}=[fill=white,draw=none,text=black]

  \node[state] 		   (wa)                     {$wa$};
  \node[state]         (nt) [above right of=wa] {$nt$};
  \node[state]         (sa) [below right of=wa] {$sa$};
  \node[state]         (gld) [right of=nt] 		{$gld$};
  \node[state]         (nsw) [below right of=gld]{$nsw$};
  \node[state]         (vc) [below of=nsw] 		{$vc$};
  \node[state]         (tas) [below of=vc] 		{$tas$};

  \path (wa) edge              node {} (nt)
             edge              node {} (sa)
        (nt) edge              node {} (sa)
             edge 			   node {} (gld)
        (sa) edge              node {} (gld)
             edge              node {} (nsw)
             edge              node {} (vc)
        (gld) edge             node {} (nsw)
        (nsw) edge 			   node {} (vc);
\end{tikzpicture}
\pagebreak

\section{Exemple satisfiable}

\formula{$./sat\_graph\ \ MiniSat\_v1.14\_linux\ \ tmp.json\ \ 3$}
\ \\
\begin{lstlisting}
==================================[MINISAT]===================================
| Conflicts |     ORIGINAL     |              LEARNT              | Progress |
|           | Clauses Literals |   Limit Clauses Literals  Lit/Cl |          |
==============================================================================
|         0 |      55      117 |      18       0        0     nan |  0.000 % |
==============================================================================
restarts              : 1
conflicts             : 2              (3101 /sec)
decisions             : 11             (17054 /sec)
propagations          : 51             (79070 /sec)
conflict literals     : 5              (0.00 % deleted)
Memory used           : 1.68 MB
CPU time              : 0.000645 s

SATISFIABLE
======== Model ========
wa prendra la couleur 1
nt prendra la couleur 2
sa prendra la couleur 0
gld prendra la couleur 1
nsw prendra la couleur 2
vc prendra la couleur 1
tas prendra la couleur 2

\end{lstlisting}\pagebreak

\section{Exemple non satisfiable}

\formula{$./sat\_graph\ \ MiniSat\_v1.14\_linux\ \ tmp.json\ \ 2$}
\ \\
\begin{lstlisting}
==================================[MINISAT]===================================
| Conflicts |     ORIGINAL     |              LEARNT              | Progress |
|           | Clauses Literals |   Limit Clauses Literals  Lit/Cl |          |
==============================================================================
|         0 |      32      64  |      10       0        0     nan |  0.000 % |
==============================================================================
restarts              : 1
conflicts             : 2              (3101 /sec)
decisions             : 1              (17054 /sec)
propagations          : 10             (79070 /sec)
conflict literals     : 1              (0.00 % deleted)
Memory used           : 1.68 MB
CPU time              : 0.000645 s

UNSATISFIABLE


\end{lstlisting}

\end{document}

