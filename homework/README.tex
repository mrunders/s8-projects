	\documentclass[french, 12pt]{report}
\usepackage[latin1, utf8]{inputenc}
\usepackage{color}
\usepackage{graphicx}
\usepackage{listings}
\usepackage{amssymb}
\usepackage{amsmath}
\usepackage{pstricks}
\usepackage{enumitem}
\usepackage{multicol}
\usepackage{verbatim}
\usepackage{listings}
\usepackage{tikz}
\usetikzlibrary{arrows,automata}
\usetikzlibrary{shapes,snakes}
\usepackage{pgfplots}
\usepackage{pgfplotstable}
\usepackage{xifthen}
\usepackage{fancyhdr}
\usepackage{caption}
 
%%\pgfplotsset{compat=1.16}
\setlist[description]{leftmargin=\parindent,labelindent=\parindent}

\definecolor{gray}{rgb}{0.4,0.4,0.4}
\definecolor{darkblue}{rgb}{0.0,0.0,0.6}
\definecolor{cyan}{rgb}{0.0,0.6,0.6}
\definecolor{darkgreen}{RGB}{0,150,0}
 	
\newcommand{\cblue}[1]{ \textcolor{blue}{#1}}
\newcommand{\corange}[1]{ \textcolor{orange}{#1}}
\newcommand{\cviolet}[1]{ \textcolor{violet}{#1}}
\newcommand{\crouge}[1]{ \textcolor{red}{#1}}
\newcommand{\cvert}[1]{ \textcolor{darkgreen}{#1}}
\newcommand{\cgris}[1]{ \textcolor{gray}{#1}}

%% ------------------------- Header Footer
\pagestyle{fancy}
\fancyhf{}
\fancyhead[LE,RO]{Master 2 IA Memo}
\fancyhead[RE,LO]{ }
\fancyfoot[CE,CO]{\leftmark}
\fancyfoot[LE,RO]{\thepage}
% -------------------------- END Header Footer

%% ------------------------- Formular
\newcommand{\cformular}[2]{
\begin{center} 
\begin{description} 
\item[#1] #2 
\end{description} 
\end{center}
}
%% ------------------------- END Formular

%% ------------------------- RO Model
\newcommand{\rovarinout}[5]{
\begin{description}
\item[La variable entrante sera] #1
\item[La variable sortante sera] #2 car:
\end{description}
\begin{multicols}{3}
#3, #4, #5
\end{multicols}
}

\newcommand{\romodel}[8]{
\begin{multicols}{2}
[Voici le nouveau modèle:]
\begin{description}
\item[Déterminer] #1
\item[#2] #3 % ((2) maximisant | minimisant )
\item[Variables hors base] #4
\item[Variables de Base] #5
\item[Solution admissible] #6 et Z = #7
\end{description}
#8 % contraintes
\end{multicols}
}
%% ------------------------- END RO Model

%% ------------------------- XML

\lstset{
  basicstyle=\ttfamily,
  columns=fullflexible,
  showstringspaces=false,
  commentstyle=\color{gray}\upshape
}

\lstdefinelanguage{XML}
{
  morestring=[b]",
  morestring=[s]{>}{<},
  morecomment=[s]{<?}{?>},
  stringstyle=\color{black},
  identifierstyle=\color{darkblue},
  keywordstyle=\color{cyan},
  morekeywords={xmlns,version,type}% list your attributes here
}
%% ------------------------ END XML

%% ------------------------ Almost all
\newcommand{\almost}{\mid\kern-0.40em{\backsim}\ }
%% ------------------------ END Almost all

%% ------------------------ Inverse DL lite
\newcommand{\inverse}{\urcorner\ }
%% ------------------------ End Inverse DL lite

%% ------------------------ Python code for Machine leaning
\definecolor{codegreen}{rgb}{0,0.6,0}
\definecolor{codegray}{rgb}{0.5,0.5,0.5}
\definecolor{codepurple}{rgb}{0.58,0,0.82}
\definecolor{backcolor}{rgb}{0.97,0.97,0.95}

\lstdefinestyle{mlpythoncode}{
    backgroundcolor=\color{backcolor},   
    commentstyle=\color{codepurple},
    keywordstyle=\color{codegreen},
    numberstyle=\tiny\color{codegray},
    stringstyle=\color{magenta},
    basicstyle=\footnotesize,
    breakatwhitespace=false,         
    breaklines=true,                 
    captionpos=b,                    
    keepspaces=true,                 
    numbers=left,                    
    numbersep=5pt,                  
    showspaces=false,                
    showstringspaces=false,
    showtabs=false,                  
    tabsize=2,   
    emph={[2]sklearn, model_selection, train_test_split,KFold, linear_model, LinearRegression, LogisticRegression,
    DecisionTreeRegressor, tree, neighbors, KNeighborsClassifier, svm, SVC, metrics, confusion_matrix,precision_recall_fscore_support,
    LeaveOneOut},
	emphstyle=[2]\color{blue}
}

\lstdefinestyle{myjson}{
    string=[s]{"}{"},
    stringstyle=\color{blue},
    comment=[l]{:},
    commentstyle=\color{black},
}

\newcommand{\sepline}{\textcolor{gray}{\noindent\rule{14cm}{0.1pt}}}
\newcommand{\paramtype}[1]{\textcolor{gray}{\textsf{\textit{#1}}}}

\newcommand{\funcdoc}[4]{
	\ \\
	\textit{\textsf{\cblue{#1}}}
    \ifthenelse{\isempty{#2}}%
    {}%
	{    \ \\\sepline\ \\
	\textbf{Variables}
	{#2}}
    \ifthenelse{\isempty{#3}}%
    {}%
	{    \ \\\sepline\ \\
	\textbf{Contraintes}
	{#3}}
    \ifthenelse{\isempty{#4}}%
    {}%
	{    \ \\\sepline\ \\
	\textbf{}
	{#4}}
}

%% ------------------------ END Python code

%% ------------------------ FORMULA
\newcommand{\formula}[1]{
\begin{center}
{#1}
\end{center}
}
%% ------------------------ END FORMULA


%% ------------------------ FORME SHAPE
\newcommand{\cshape}[2]{
\begin{center}
\scalebox{#1}{#2}
\end{center}
}
%% ----------------------- END FORME SHAPE

\title{CNF informations}
\author{LAURENT Thomas}
\date{Master 2 informatique 2019}

\begin{document}
\maketitle
\pagebreak

\section{Ce que contient l'archive}

\begin{description}
\item[\_\_main\_\_.py] Le programme qui va traiter le fichier en format DIMACS.
\item[dimacs.in] Un fichier au format dimacs contenant le cnf de test.
\item[README.pdf] Coucou c'est moi.
\end{description}

\section{Utilisation du script}
Voici un exemple:
\formula{$python \_\_main\_\_.py <fichier dimacs>$}

\begin{description}
\item[fichier dimacs] et le répertoire vers le fichier à traiter
\end{description}

\pagebreak

\section{Procédures complexes}

Au tout début le fichier est lue lignes par lignes, si celle ci contient (en premier caractère):
\begin{description}
\item[$c$] on l'ignore
\item[$p$] on casse la ligne (en utilisant les espaces en tant que délimiteur) puis on récupère les deux derniers éléments (nb vars, nb lines)
\item[$sinon$] on casse la ligne (en utilisant les espaces en tant que délimiteur) en les cast en $Integer$ le tout dans une liste
\end{description}

A l'issue de la lecture nous avons une matrice représentant la formule cnf, le nombre, le nombre de variables puis le nombre de lignes.

\subsection{détection des 2cnf (binaire)}
une $nbcnf2$ (nom de la variable dans le code) est donné par l'instruction:
\begin{center}
\lstset{style=mlpythoncode}
\begin{lstlisting}[language=Python]
len(list(filter(lambda x : len(x) == 2, self.cnf)))
\end{lstlisting}
\end{center}

\begin{description}
\item[$self.cnf$] est la matrice.
\end{description}

nous effectuons un filtre sur la matrice pour ne garder que les sous listes de tailles 2. Puis nous récupérons la taille de la liste une fois filtré.\\

\pagebreak
\subsection{détection des horn}
une $nbhorn$ (nom de la variable dans le code) est donné par l'instruction:

\begin{center}
\lstset{style=mlpythoncode}
\begin{lstlisting}[language=Python]
len(list(filter(lambda x : len(x) == 1, [list(filter(lambda x : x > 0, sl)) for sl in self.cnf])))
\end{lstlisting}
\end{center}

\begin{description}
\item[$self.cnf$] est la matrice.
\end{description}

dans un premier temps:
\begin{center}
\lstset{style=mlpythoncode}
\begin{lstlisting}[language=Python]
[list(filter(lambda x : x > 0, sl)) for sl in self.cnf]
\end{lstlisting}
\end{center}

nous allons réduire la matrice en une autre matrice contenant que des littéraux positif. (que nous allons nommer $M$ dans la suite):
\begin{center}
\lstset{style=mlpythoncode}
\begin{lstlisting}[language=Python]
len(list(filter(lambda x : len(x) == 1, M)))
\end{lstlisting}
\end{center}

Puis nous filtrons les clauses en ne gardant que celle qui contienne que 1 élément, Puis nous récupérons la taille de la liste une fois filtré.\\

\subsection{détection des reverse horn}
une $nbreverse_horn$ (nom de la variable dans le code) est donné par l'instruction:

\begin{center}
\lstset{style=mlpythoncode}
\begin{lstlisting}[language=Python]
len(list(filter(lambda x : len(x) == 1, [list(filter(lambda x : x < 0, sl)) for sl in self.cnf])))
\end{lstlisting}
\end{center}

\begin{description}
\item[$self.cnf$] est la matrice.
\end{description}

Le fonctionnement est identique que la détection des horn sauf que l'on ne garde que les littéraux négatif.

\end{document}