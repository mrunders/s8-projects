
\section{Sans intelligence artificiel}
On parle du premier OCR créé dans le monde en parlant des travaux de l'américain \cauthors{Charles R Carey} en 1870 qui inventa un scanneur rétinien appliquant des patterns en mosaïque et étant appliqué sur une image envoyé en entré.\\
\\
Dans certaines littératures on parle du premier OCR en 1900 par un russe du nom de \cauthors{Tyurin} qui a tenté d'aider sur des travaux de reconnaissance de symbole pour les personnes handicapé.\\
\\
La première machine OCR a était crée en 1929 par un ingénieur Australien du nom de \cauthors{Gustav Tauschek}, $'the\ reading\ machine'$, le processus est fait en lisant le caractère avec une source lumineuse puis de comparer le caractère à une suite d'images de référence.\\
\\
On parle du premier OCR automatique en 1940 lors de l'introduction à l'air digital (4 ans après la machine de \cauthors{Turing}), dans un premier temps  les travaux d'automatisation OCR ont était réalisé directement des caractères issue d'une machine ou ont était réalisé via un petit ensemble de papier manuscrit où les caractères était finement bien représenté et distingué des autres. La conversion des papiers en binaire était faible, en effet l'extraction des caractères en format vectoriel était une procédure légère.\\
\\
L'ORC a commencé son assertion lorsque les entreprisses avaient un besoin de convertir les rapports de vente en carte perforé pour le $'data\ processing'$.\\
\\
En 1966 l'université Rochester d'IBM développa le premier $'Handwriting\ scanner'$ capable de lire n'importe quel nombres écrit à la main.\\
\\
En 1968, un Suisse du nom d'Adriant Frutiger a introduit deux nouvelles classes de police de caractères nommé $OCR-A$ et $OCR-B$ qui sont tout les deux un dérivé de le police de caractères $Sans-serif$ et étant deux type de polices pouvant être facilement reconnue par un ordinateur ou pouvant être facilement compréhensible pas un être humain. L'organisation international de la normalisation ont nommé $OCR-A$ comme étant la norme $ISO 1073-1:1976$ et $OCR-B$ du nom de norme $ISO 1073-2:1976$.
\\\\\\\\
1974 vient $Omni-font$, une production de $Ray Kurzweil$, un logiciel pouvant reconnaître n'importe quel caractère imprimé peu import la police de caractères utilisé.\\
\\

\section{Avec des technique de machine learning}
En 2005 le premier moteur gratuit open source nommé $tesseract-ocr$. En 2008 la famille de logiciels $Adobe Acrobate$ ont inclue l'outil OCR dans leurs fichiers pdf.\\
\\
Finalement le premier OCR implémentant du machine learning fut crée en 2013 par l'institut national des standards et de la technologie ($MNIST$) utilisant une très vaste base de données de digits écrit par l'humain.\\
\\
Pour finir $Google$ a offert un support OCR sur tout leur fichiers $google drive$.\\
\\

\pagebreak