\section{Problématique}

La reconnaissance des caractères est un problème de reconnaissance de patterns, dans une image l'OCR doit savoir distinguer un ensemble de formes formant des probables caractères. L'effort est de pouvoir analyser l'écriture humaine comportant beaucoup de différences d'un individu à un autre, la taille du caractère, sa longueur ou hauteur joue dans la reconnaissance de probable faux positif.\\ 
Prenons deux technique d'écriture, l'une étant l'écriture lié qu'on apprend à l'école et l'autre l'écriture que je vais appelé espacé qui reprend le format des caractères affiché sur l'écran d'un ordinateur. les deux textes ayant la même interprétation pour l'humain qui l'ai lit, l'OCR peut tout de même donner deux résultats différent lors du processus de transformation.\\
Comment l'intelligence artificiel a pu trouver une solution à ce problème via l'apprentissage et les réseaux de neurones.

\pagebreak