
\chapter{l'écriture d'avant à maintenant}
Depuis de nombreux siècles, l'Homme a su écrire des textes dans bien de nombreux langages.\\
Selon certain récits, l'écriture serait né en Mésopotamie durant l'an 3300 avant Jésus christ et étant composé de signes ou de pictogrammes représentant des mots ou des concepts. Ses textes était gravé dans la pierres, taillé dans le bois, écrit avec des roseaux taillées en pointes sur les murs des cavernes ou plus tard écrit sur du papier (ou des végétaux).\\
C'est ainsi que la connaissance et le patrimoine pu se conserver et être apprit par d'autres sociétés via dans un première instant la ré-écriture des textes. La ré-écriture n'étant jamais parfaite, la forme des caractères étaient différents d'un endroit à un autre, ce genre de transports ont notamment donné les chiffres tel qu'on l'ai connait aujourd'hui.\\
Depuis cette époque, de nombreux documents (de tout type) ont vue le jour, et même à ce jour nous continuons d'en créer.\\
Depuis la digitalisation et le format dématérialisé, l'informatique joue une part importante dans le stockage et le partagent de l'information, mais néanmoins la forme matérialisé des document ont une durée de vie limité, sur certaines surface l'ancre peut disparaitre ou la nature qui peut effacer certaines frises ou gravures, la restauration de ses documents est un processus utilisé mais qui a une limite.\\
L'invention du numériseur a permit de digitaliser certains document comme des anciens livres ou des documents officiel.
Les documents officiel sont des pièces qui ont besoin d'être validé pour prouver leurs authenticité, mais nous ne pouvons pas simplement envoyer une image d'une pièce d'identité à une personne qui s'occupe de vérifier son authenticité.
Pour vérifier qu'une série de chiffres est valide, l'assistante devrait lire la série de chiffres recopier dans une base de donné, mais cette opération de peut être fait manuellement tout en conservant une vitesse de traitement rapide.\\
D'où l'informatique où l'intelligence artificiel donne une solution d'automatisation du traitement des images et de la récupération de données.

\pagebreak